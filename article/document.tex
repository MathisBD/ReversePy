\documentclass[french]{article}
\usepackage[T1]{fontenc}
\usepackage[utf8]{inputenc}
\usepackage{lmodern}
\usepackage[a4paper]{geometry}
\usepackage{babel}

\title{Title bla blah}

\author{Mathis Bouverot-Dupuis}
\date{June \& July 2021}

\begin{document}

\maketitle 

\begin{abstract}
	
\end{abstract}


\iffalse

S'inspirer du papier des Bardins.

Détailler un peu la démarche.

Intro : after context, tell how my work is different : shallow description of the VM (using only the execution trace)

1. Recognizing the VM execution
	-> cfg : (with some pictures)
		-> how I chose to build the cfg
		-> first try : we can see the computed gotos
		-> second try : centralized fetch
	-> extracting traces for each opcode (pin tool)
		-> what is a trace precisely (where do we break it) ?
		-> maybe an example for a really short trace (e.g. I think jump_absolute is ~20 lines) : highlight fetch/dispatch/opcode
		
2. Abstract VM model -> at the start of the article : describe informally a VM
	-> stack VM (no registers)
	-> stack : contiguous 8byte cells, pointer to the top of stack
	-> code block : contiguous 2byte cells, opc + arg, instr pointer
	-> frame : contains a code block + stack, frame pointer
	
3. extracting the VM state/recognizing the VM model (pointers, instr blocks, frame changes) -> merge with 2., and 
	-> list the characteristics of each pointer (we look at register values between opcodes, i.e. just before the fetch instruction)
		ip :
			alignment >= 0x2
			more than a few distinct values
			never overwritten (explain)
			max instr streak (explain, give offsets) >= a little number
		sp :
			alignment >= 0x8
			more than a few distinct values
			is overwritten at least a few times
			max stack streak (explain, give offsets) >= a little number
				-> explain why we exclude 0 from the offsets (to distinguish sp and fp)
	-> enough for sp, not for ip : 
		heuristic : 
			-> sp gives us the frame changes (look for big sp changes) 
			-> code block changes at most frame changes (except recursive function calls) 
			-> we expect big ip jumps at frame changes
	-> fp : we expect it to change at frame changes
		-> we find 2 that change *exactly* at the same time as sp : one is esp
	-> method for finding code blocks
		-> put instrs in the same block if :
			either close in memory
			or executed one after the other without a frame change
			
4. 'shallow' semantics of opcodes
	-> what kind of semantics do we want ? 
		-> only look at VM state between opcodes : don't look at the asm trace (implementation) of each opcode
		-> variations of sp :
			sp <- sp + cste 
			sp <- sp - 8*arg + cste (arg tells how many values to 'eat' on the stack)
		-> variations of ip :
			ip <- ip + cste (go to next instr)
			ip <- ip + arg + cste (relative jump)
			ip <- code-block-start + arg + cste (absolute jump)
			ip <- either absolute jump or next instr (conditional jump)
		-> we require cste to be small (a few stack/code cells at most)
		
5. experimental setup 
	-> link to github
	-> version of pin & python, a word on the centralized fetch compile option, say how it breaks with different python versions (e.g. libpython on Alexandre's machine) 
	-> IMPORTANT (maybe ?) : code might break on big-endian machines (e.g. when encoding/decoding json values)
	-> brief description of the code organization -> maybe add a readme in the github repo instead ?
	
\fi 

\end{document}
